\documentclass[11pt, oneside]{article}   	% use "amsart" instead of "article" for AMSLaTeX format
\usepackage{geometry}                		% See geometry.pdf to learn the layout options. There are lots.
\geometry{letterpaper}                   		% ... or a4paper or a5paper or ... 
\usepackage{graphicx}				% Use pdf, png, jpg, or eps§ with pdflatex; use eps in DVI mode
								% TeX will automatically convert eps --> pdf in pdflatex		
\usepackage{amssymb}
\usepackage{program}

\title{Assignment 2: Classical and Local Search, Adversarial Search, Constraint Satisfaction Problems, and Logic}
\author{Alex Smirnov, Scott Reyes}
\date{March 12, 2017}							% Activate to display a given date or no date

\begin{document}
\maketitle
%\section{}
%\subsection{}
\begin{flushleft}

\section*{Problem 1:}
\begin{center}
\begin{tabular}{ |c|c|c|} 
\hline
Number of Variables & Median Running Time (s) & Average bit flips \\\hline
20 & 41.994 & 109566 \\\hline
50 & 299.201 & 761892 \\\hline
75 & 681.720 & 1621230 \\\hline
100 & 981.720 & 3117231 \\
\hline
\end{tabular}
\end{center}

The number of variables in each instance has a strong effect on the median running time and average bit flips performed. For the 20 variable problems, the comupter would solve 50 of them in just a couple on minutes. As we tested the 50 and 75 variable problems, the average time per problem increased significantly, as did the amount of bit flips. The time and bit flips were almost 7 times higher for the 50 variable problems, and 16 times higher for the 75 variable problems. Finally, the 100 variable problems took so much time to run, that we had to test them separately instead of all together like the others. They were almost 20 times higher in median running time and bit flips. \\\medskip
The performance becomes much slower with more variables in the problem because the computer has to store and evaluate many more values. The bit flip amount increased significantly because each one of the variables would contribute to extra bit flips every cycle. Finally, the amount of clauses increased as the amount of variables increased, so evaluating all of them became much more time consuming.\\\medskip

\section*{Problem 2:}


Problem 2: \\
$a$. Intermediate Node Values:\\
A = Start Node \\
B = 6, C = 17 \\
D = 8, E = 6, F = 22, G = 17 \\
H = 8, I = 5, J = 6, K = 6, L = 18, M = 22, N = 8, O = 17 \\\medskip

$b$. Alpha-Beta Pruning - Nodes Not Considered:\\
Node I's right child, 16\\
Node N's right child, 8\\
Node G's right children, O, 90, and 17\\\medskip
Alpha Beta Values:\\
A: $\alpha = 6, \beta = 8$\\
B: $\alpha = 6, \beta = 8$ \\
C: $\alpha = 18, \beta = 22$ \\
D: $\alpha = 8, \beta = +\infty$ \\
E: $\alpha = 6, \beta = 8$ \\
F: $\alpha = 18, \beta = +\infty$ \\
G: $\alpha = 18, \beta = 8$ \\
H: $\alpha = -\infty, \beta = 8$ \\
I: $\alpha = 8, \beta = 5$ - pruned right subtree\\
J: $\alpha = 6, \beta = 8$ \\
K: $\alpha = 6, \beta = 8$ \\
L: $\alpha = -\infty, \beta = 18$ \\
M: $\alpha = 18, \beta = 22$ \\
N: $\alpha = 18, \beta = 8$ - pruned right subtree\\
O: pruned \\\medskip

$c$. The MAX player will choose to go right at the root state, to the C node. The MAX player will also go right at the root node using the alpha-beta pruning algorithm. The best move computed by the two versions is guaranteed to be the same, because alpha-beta pruning only removes nodes that wouldn't be visited anyway.\\\medskip

$d$. The heuristic-aided minimax algorithm reorders the nodes by left aligning the min values for the MIN player, and the max values for the MAX player to place the most likely path of the algorithm to the left and the least likely to the right.\\
New Tree: \\
A = root\\
B = 17, C = 6\\
D = 17, E = 22, F = 6, G = 8\\
H = 17, I = 8, J = 22, K = 18, L = 6, M = 6, N = 8, O = 5\\
17, 90, 8, 99, 22, 28, 18, 18, 6, 15, 6, 83, 8, 99, 5, 16 \\\medskip

The nodes not examined by the alpha-beta pruning will increase from 5 to 13, with nodes K, G, N, and their children getting pruned. The right children of nodes I, L and M are also pruned.\\\medskip

$e$. If the opponent plays randomly, the min nodes are picked arbitrarily at every MIN level, causing changes to happen upward through the tree all the way to the root. \\\medskip

Min Replaced With Random:\\
A = root node\\
B = 99, C = 22\\
D = 99, E = 83 \\
F = 22, G = 99 \\
H = 99, I = 5, J = 83, K = 6, L = 18, H = 22, N = 99, O = 17 \\\medskip
In this case, the MAX player will choose to go left instead of right, because due to the random selections made by the opponent, the left tree has much better options now instead of the right. You cannot apply alpha-beta pruning in this case because the opponent moves randomly, so the moves cannot be predicted fully. If we don't know the exact values of alpha and beta, we cannot perform the algorithm.\\\medskip


\section*{Problem 3:}
\subsection*{a}
For solving a Sudoku like this, the simplest way would be to have 9 arrays as each row to hold the numbers where each cell holds a number from 1-9. The constraints are that a single row must not have a duplicate number, cells in the same index in all rows must not have a duplicate number, cells that integer divide by 3 to the same number must not have the same number and certain numbers are fixed in certain cells.
\subsection*{b}
The start state is the current instance of the values in the board. The successor function is a function that goes through the board, finds the space with the lowest number of conflicts on placement of a value, and assigns the cell the corresponding number. To test the goal, each cell can be checked to see if it has a conflict. The path cost would be the number of conflicts a cell has given the placement of a character.
\subsection*{c}
An easy Sudoku problem has fewer blocks with a high branching factor after checking the constraints while a harder problem has more blocks with a higher branching factor. Having a high branching factor after checking constraints means that you have a higher chance to pick a wrong choice which may not be realized until much farther in computation. Once there, you would have to backtrack all the way to that first wrong choice to find out what is the correct answer and restart the computation from there.
\subsection*{d}
It may be the case that a genetic algorithm would be better than a backtracking search. The algorithm would essentially take these steps:

\begin{program}
\BEGIN \\
\PROC |Selection|(|State Array|)\BODY \

\FOR i:=0 \TO |State Array|.length \DO
numCorrect:=0
\FOR j:=0 \TO StateArray_{i}.length \DO
numCorrect:= |findNumCorrectForCell|()
\OD
|ResultArray|.insertByNumCorrect(StateArray_{i},numCorrect);
\OD
trim(ResultArray, NumberOfSamples)
return\ |ResultArray|;
\
\ENDPROC

\PROC |Crossover|()\BODY \

\FOR i := 0;\TO |State Array|.length\DO
\FOR j := 0;\TO StateArray_{i}.length\DO
\IF Random > .50\THEN swap(StateArray_{ij}.number,StateArray_{(i+1)j}.number )
\OD
i+=2;
\OD
\
\ENDPROC

\PROC |Mutation|()\BODY \

\FOR i = StartingIndexOfNotElites;\TO |State Array|.length\DO
\FOR j = 0\TO StateArray_{i}.length\DO
\IF(Random > .5)\THEN StateArray_{ij} = RandomNumber;
\OD
\OD
\ENDPROC
\PROC |SudokuGene|()\BODY \

|State Array|
  \FOR i:=1 \TO MaxIterations \DO
     elites = |topElites|(|State Array|); \\
     \IF |areSatisfied(elites)| \THEN |return|\ elites_{1}\ \FI
     |Selection|(|State Array|)
     |Crossover|(|State Array|)
     |Mutation|(|State Array|)
     \OD
\ENDPROC
\end{program}
\end{flushleft}
\end{document}  
